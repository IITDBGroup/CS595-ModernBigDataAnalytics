\documentclass[12pt]{article}
\textwidth=7in
\textheight=9.5in
\topmargin=-1in
\headheight=0in
\headsep=.5in
\hoffset  -.85in

\usepackage{color}
\usepackage{hyperref}
\pagestyle{empty}

\renewcommand{\thefootnote}{\fnsymbol{footnote}}
\begin{document}

\begin{center}
{\bf \Large CS 595-xx - Topics in Modern Big Data Analytics}\\
\end{center}

\setlength{\unitlength}{1in}

\begin{picture}(6,.1)
\put(0,0) {\line(1,0){6.25}}
\end{picture}

\renewcommand{\arraystretch}{2}
%%%%%%%%%%%%%%%%%%%%%%%%%%%%%%%%%%%%%%%%%%%%%%%%%%%%%%%%%%%%
\vskip.25in


%%%%%%%%%%%%%%%%%%%%%%%%%%%%%%%%%%%%%%%%%%%%%%%%%%%%%%%%%%%%
\vspace*{.15in}

{\large \noindent \textbf{Course Description:} }

Big data technologies, in particular, scalable distributed platforms for storage
and analytics enable processing of massive datasets for analytics, machine
learning, and other use cases.  This course provides a comprehensive overview of
algorithms, systems, and techniques for Big Data processing. In a semester-long
project, students will extend existing big data platforms.  Additionally, in the
seminar component of this course we will discuss cutting edge research and
industrial developments in the field.

%%%%%%%%%%%%%%%%%%%%%%%%%%%%%%%%%%%%%%%%%%%%%%%%%%%%%%%%%%%%
\vskip.25in
{\large \noindent\textbf{Course Material:}  }

\noindent The following text book will be helpful for following the course and studying the presented material.\\

{\footnotesize
  \noindent White, \textbf{Hadoop: The Definitive Guide}, 4th Edition, O'Reilly Media, 2015\\
}

\noindent One of the following standard text books on databases in general may be helpful. However, this is not required reading material.\\

{\footnotesize
\noindent  Elmasri and Navathe. \textbf{Fundamentals of Database Systems}, 6th Edition , Addison-Wesley , 2003 \\
\noindent  Ramakrishnan and Gehrke. \textbf{Database Management Systems},  3nd Edition ,  McGraw-Hill , 2002 \\
\noindent  Silberschatz, Korth, and Sudarshan. \textbf{Database System Concepts}, 6th Edition , McGraw Hill , 2010\\
\noindent  Garcia-Molina, Ullman,  and  Widom. \textbf{Database Systems: The Complete Book}, 2nd Edition, Prentice Hall, 2008}\\[3pt]

\noindent Slides for the course will be made available on the course webpage.


%%%%%%%%%%%%%%%%%%%%%%%%%%%%%%%%%%%%%%%%%%%%%%%%%%%%%%%%%%%%
\vskip.25in
{\large \noindent\textbf{Prerequisites:}}\\

No formal prerequisites, but some background in databases and/or distribute programming is useful.
% \begin{itemize}
% \item \textit{Courses:} CS425
% \end{itemize}

\pagebreak
%%%%%%%%%%%%%%%%%%%%%%%%%%%%%%%%%%%%%%%%%%%%%%%%%%%%%%%%%%%%
\vskip.25in
{\large \noindent \textbf{Course Details:} }

\noindent The following topics will be covered in the course:

\begin{itemize}
\item \textbf{Foundations of Scalable and Distributed Storage and Computation}
  \begin{itemize}
  \item Fault tolerance
  \item Eventual consistency and consensus protocols
  \item Load balancing
  \item Scalable algorithm design
  \item Data placement techniques
  \end{itemize}
\item \textbf{Distributed Storage}
  \begin{itemize}
  \item Distributed file systems and replication
  \item Key-value and distributed document Stores
  \item Structured distributed storage solutions
  \end{itemize}
\item \textbf{Distributed Batch Processing}
  \begin{itemize}
  \item Specifying computations as dataflows
  \item DISC systems
  \item Iterative and incremental dataflows
  \end{itemize}
\item \textbf{High-level Dataflow Languages}
\begin{itemize}
\item Scripting and query languages
\item Graph processing
\end{itemize}
\item \textbf{Streaming Analytics}
  \begin{itemize}
  \item Distributed stream processing
  \item Publish-subscribe systems
  \end{itemize}
\item \textbf{Distributed Transaction Processing}
  \begin{itemize}
  \item The 2PC protocol
  \item Transaction processing over partitioned storage
  \end{itemize}
\end{itemize}

\pagebreak
%%%%%%%%%%%%%%%%%%%%%%%%%%%%%%%%%%%%%%%%%%%%%%%%%%%%%%%%%%%%
\vskip.25in
{\large \noindent \textbf{Workload} }

\noindent The workload will consist of
\begin{enumerate}
\item A semester long project related to extending an existing Big Data platform
\item Review a research paper related to state-of-the-art techniques in Big Data processing and present it in the course
\end{enumerate}


%%%%%%%%%%%%%%%%%%%%%%%%%%%%%%%%%%%%%%%%%%%%%%%%%%%%%%%%%%%%
\vskip.25in
\noindent {\large \textbf{Course Objectives}:}

\noindent After attending the course students should:

\begin{itemize}
\item Understand the challenges of processing queries and other data-intensive computations in a distributed fashion
\item Be familiar with scalable storage and compute solutions; understand their benefits and limitations
\item Learn about different types of scalable systems including \ldots
  \begin{itemize}
  \item \textit{Distributed file systems}
  \item Scalable storage techniques such as \textit{key-value stores} and distributed structured storage solutions such as \textit{HBase}
  \item DISC platforms such as MapReduce, Spark, and Flink
  \item Specialized systems for, e.g., \textit{graph data} such as Giraph and support for graph data in general purpose DISC platforms
  \item \textit{Publish-subscribe systems} such as Kafka
  \item Distributed transaction processing systems
  \end{itemize}
\item Understand what \textit{fault tolerance} is and how it can be achieved through replication, logical logging (as in Spark), and through \textit{consensus protocols} like Paxos and Raft
\item Understand how \textit{load-balancing} is achieved in DISC systems
\item Understand \textit{data placement techniques} including horizontal and vertical partitioning and how they utilized by DISC frameworks
\item Learn about the distributed algorithms employed by DISC platforms for implementing the higher-order functions exposed to the user
\end{itemize}

%%%%%%%%%%%%%%%%%%%%%%%%%%%%%%%%%%%%%%%%%%%%%%%%%%%%%%%%%%%%
\vskip.25in
\noindent {\large \textbf{Grading Policy}:}

The grading scheme is as follows:

\begin{itemize}
\item A: 80\% or higher
\item B: 50\% or higher
\item C: 35\% or higher
\item E: below 35\%
\end{itemize}

The weighting of the individual components are:

\begin{itemize}
\item Programming Project: 50\%
\item Literature Review: 50\%
\end{itemize}

\noindent {\large \textbf{Illinois Tech’s Sexual Harassment and Discrimination Information}:}

\begin{itemize}
\item Sexual harassment, sexual misconduct, and gender discrimination by any
  member of the Illinois Tech community is prohibited. This includes harassment
  among students, staff, or faculty. Sexual harassment by a faculty member or
  teaching assistant of a student over whom they have authority or by a
  supervisor of a member of the faculty or staff is particularly serious. Such
  conduct may easily create an intimidating, hostile, or offensive environment.

\item  Illinois Tech encourages anyone experiencing sexual harassment or sexual misconduct to speak with the Title IX Office for information on the resolution process and support options.

\item  You can file a complaint electronically at \url{http://iit.edu/incidentreport}, which may be completed anonymously. You may also file a complaint in-person by contacting the Title IX Coordinator, Virginia Foster at \url{312-567-5725} / \url{mailto:foster@iit.edu} or the Deputy Title IX Coordinator \url{312-567-5726} / \url{mailto:eespeland@iit.edu}.

\item If you are not ready to file a formal complaint but wish to learn about your rights and options, you may contact Illinois Tech’s Confidential Advisor service at \url{773-907-1062}. You can also contact a licensed practitioner in Illinois Tech’s Student Health and Wellness Center at \url{312-567-7550}

\item For a comprehensive list of resources regarding counseling services, medical assistance, legal assistance and visa and immigration services, you can visit the Title IX Office’s website at \url{https://web.iit.edu/hea/resources}
\end{itemize}
\end{document}